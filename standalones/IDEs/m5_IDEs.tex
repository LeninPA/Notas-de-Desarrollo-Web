\documentclass[11pt]{article}

%%%%%%%%%%%%%%%%%%%%%%%%%%%%%%%%%%%%%%%%%%%%%%%%%%
% Este es el mismo código que en:
% mod5.tex
% Recomendaciones de IDEs
% Adaptado para que la tabla se viera en una sola página
%%%%%%%%%%%%%%%%%%%%%%%%%%%%%%%%%%%%%%%%%%%%%%%%%%


% Datos del documento
\title{Recomendaciones de IDEs}
\author{Equipo del Curso Web 2021}
\date{}
% Paquetes
\usepackage[subpreambles=true]{standalone}	% Apoya en la importación de documentos
\usepackage{import}													% Permite la importación de documentos
\usepackage{float} 													% For new floating environments
\usepackage{epsfig,graphicx,subfigure} 			% For inserting figures
	\graphicspath{ {img/} }
\usepackage[utf8]{inputenc}									% Símbolos
\usepackage[spanish]{babel}									% Idiomas
	\decimalpoint															% Cambia , por . en los decimales
\usepackage{multirow} 											% For merging cells in tables
\usepackage{array,amssymb} 							% For multi-page tables
\usepackage{longtable, makecell} 							% For multi-page tables
\usepackage{lmodern} 												% Fuente compatible
\usepackage{listings}												% Coloración de códigos
	\renewcommand\lstlistingname{Código}
	\renewcommand\lstlistlistingname{Códigos}
\usepackage[dvipsnames]{xcolor}[table]			% Colores
	\definecolor{codegreen}{rgb}{0,0.6,0}
	\definecolor{codegray}{rgb}{0.5,0.5,0.5}
	\definecolor{codepurple}{rgb}{0.58,0,0.82}
	\definecolor{backcolour}{rgb}{0.95,0.95,0.92}
\usepackage{setspace} 											% For adjusting line spacing
\usepackage{boxedminipage,fancybox} 				% For boxed texts
\usepackage{url} 	 													% For citing URL
\usepackage{imakeidx}												% For generating index
\makeindex[intoc]														% make file(s) *.idx
\usepackage{marginnote}											% Notas al margen
\usepackage{csquotes}												% Citas en línea y alejadas
\usepackage{cancel}
\usepackage{enumitem}[shortlabels]					% Para personalizar listas
\usepackage{rotating} 											% For rotating a page (landscape) or inclined texts
\usepackage{dirtytalk}																% Citas sencillas
\usepackage{tikz}														% TikZ environment
\usetikzlibrary{babel}					
\usetikzlibrary{intersections, calc, shapes.geometric}		%Permite el uso de líneas guía y sus intersecciones 
\usepackage{hyperref}										% Para referencias 
\hypersetup{
    colorlinks=true,										% Define si los \ref se colorean
    linkcolor=blue,											% Define el color de los \ref
    filecolor=magenta,									% Define el color de las referencias a documentos
    urlcolor=cyan,											% Define el color de las referencias a urls
    bookmarks=true,											% Enlaza la tabla de contenidos en el visor de pdf
    pdftitle={Apuntes de Desarrollo Web},		% Es el título del pdf que recibe el visor
    breaklinks=true,										% Permite que los links ocupen más de un renglón
    linktocpage=true										% Hace que las páginas, en vez del título de las secciones sean enlazadas
}
\urlstyle{same}

% Definición del estilo de los códigos
\lstdefinestyle{mystyle}{
    backgroundcolor=\color{backcolour},   
    commentstyle=\color{codegreen},
    keywordstyle=\color{magenta},
    numberstyle=\tiny\color{codegray},
    stringstyle=\color{codepurple},
    basicstyle=\ttfamily\footnotesize,
    breakatwhitespace=false,         
    breaklines=true,                 
    captionpos=b,                    
    keepspaces=true,                 
    numbers=left,                    
    numbersep=5pt,                  
    showspaces=false,                
    showstringspaces=false,
    showtabs=false,                  
    tabsize=2
}
\lstdefinestyle{table}{
    basicstyle=\ttfamily\footnotesize,
    breakatwhitespace=false,         
    breaklines=true,              
    tabsize=2
}

\renewcommand{\familydefault}{\sfdefault}	% Fuente

\begin{document}
    \begin{center}
        {\Huge Recomendaciones de IDEs}\\
        \vspace{0.5cm}
        {\LARGE Equipo del Curso Web 2021}
    \end{center}
    \begin{table}[!hbt]
        % \caption{Recomendaciones de IDEs}
        \begin{center}
            \begin{tabular}{c c c}
                \textbf{Nombre}	&	\textbf{Descripción}	&	\textbf{Es el indicado si...}\\
                \hline
                % Atom
                    \parbox[c]{2cm}{
                        \centering
                        \href{https://atom.io}{Atom}
                        \newline\newline
                        \includegraphics[height=1.5cm]{logo_atom}	
                    }	
                    &
                    \parbox[c]{3cm}{
                        Editor de código multiplataforma creado por GitHub
                    }
                    &	
                    \parbox[c]{8cm}{
                        \begin{itemize}
                            \item Necesitas algo ligero
                            \item Quieres algo sencillo de usar y aprender
                            \item Quieres trabajar colaborativamente
                            \item Quieres personalizar tu IDE
                        \end{itemize}
                    }
                    \\\hline
                % VSC
                    \parbox[c]{2cm}{
                        \centering
                        \href{https://code.visualstudio.com/download}{VS Code}
                        \newline\newline
                        \includegraphics[height=1.5cm]{logo_vscode}	
                    }
                    &
                    \parbox[c]{3cm}{
                        IDE multiplataforma creado por Microsoft
                    }
                    &
                    \parbox[c]{8cm}{
                        \begin{itemize}
                            \item Vas a trabajar en proyectos grandes
                            \item Quieres trabajar con varios lenguajes
                            \item Quieres personalizar tu IDE
                            \item Quieres usar varias extensiones
                        \end{itemize}
                    }
                    \\\hline
                % JetBrains
                    \parbox[c]{2cm}{
                        \centering
                        \href{https://www.jetbrains.com}{JetBrains}
                        \newline\newline
                        \includegraphics[height=1.5cm]{logo_jetbrains}	
                    }	
                    &	
                    \parbox[c]{3cm}{
                        Es una familia de IDEs especializados con versiones libres y de pago como 
                        \href{https://www.jetbrains.com/es-es/phpstorm/}{PHPstorm} o 
                        \href{https://www.jetbrains.com/es-es/webstorm/}{WebStorm}
                    }
                    &
                    \parbox[c]{8cm}{
                        \begin{itemize}
                            \item Vas a trabajar en un área especializada, e.g.: puro backend o puro frontend
                            \item Tienes una máquina con buen rendimiento
                            \item Quieres ver tus errores en tiempo real % corregir
                        \end{itemize}
                    }
                    \\\hline
                % Repl.it
                    \parbox[c]{2cm}{
                        \centering
                        \href{https://replit.com}{Repl.it}
                        \newline\newline
                        \includegraphics[height=1.5cm]{logo_replit}	
                    }	
                    &
                    \parbox[c]{3cm}{
                        IDE en línea colaborativo inspirado en Google Docs
                    }
                    &
                    \parbox[c]{8cm}{
                        \begin{itemize}
                            \item No puedes instalar programas
                            \item Requieres colaboración eficiente en tiempo real
                            \item Quieres todas las ventajas de un IDE sin necesidad de configurarlo
                        \end{itemize}
                    }
                    \\\hline
            \end{tabular}
        \end{center}
    \end{table}
\end{document}
