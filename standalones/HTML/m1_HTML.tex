\documentclass[11pt]{article}

%%%%%%%%%%%%%%%%%%%%%%%%%%%%%%%%%%%%%%%%%%%%%%%%%%
% Este es el mismo código que en:
% mod1.tex
% Etiquetas HTML
% Adaptado para que la tabla se viera en una sola página
%%%%%%%%%%%%%%%%%%%%%%%%%%%%%%%%%%%%%%%%%%%%%%%%%%


% Datos del documento
\title{Etiquetas HTML}
\author{Equipo del Curso Web 2021}
\date{}
% Paquetes
\usepackage[subpreambles=true]{standalone}	% Apoya en la importación de documentos
\usepackage{import}													% Permite la importación de documentos
\usepackage{float} 													% For new floating environments
\usepackage{epsfig,graphicx,subfigure} 			% For inserting figures
	\graphicspath{ {img/} }
\usepackage[utf8]{inputenc}									% Símbolos
\usepackage[spanish]{babel}									% Idiomas
	\decimalpoint															% Cambia , por . en los decimales
\usepackage{multirow} 											% For merging cells in tables
\usepackage{array,amssymb} 							% For multi-page tables
\usepackage{longtable, makecell} 							% For multi-page tables
\usepackage{lmodern} 												% Fuente compatible
\usepackage{listings}												% Coloración de códigos
	\renewcommand\lstlistingname{Código}
	\renewcommand\lstlistlistingname{Códigos}
\usepackage[dvipsnames]{xcolor}[table]			% Colores
	\definecolor{codegreen}{rgb}{0,0.6,0}
	\definecolor{codegray}{rgb}{0.5,0.5,0.5}
	\definecolor{codepurple}{rgb}{0.58,0,0.82}
	\definecolor{backcolour}{rgb}{0.95,0.95,0.92}
\usepackage{setspace} 											% For adjusting line spacing
\usepackage{boxedminipage,fancybox} 				% For boxed texts
\usepackage{url} 	 													% For citing URL
\usepackage{imakeidx}												% For generating index
\makeindex[intoc]														% make file(s) *.idx
\usepackage{marginnote}											% Notas al margen
\usepackage{csquotes}												% Citas en línea y alejadas
\usepackage{cancel}
\usepackage{enumitem}[shortlabels]					% Para personalizar listas
\usepackage{rotating} 											% For rotating a page (landscape) or inclined texts
\usepackage{dirtytalk}																% Citas sencillas
\usepackage{tikz}														% TikZ environment
\usetikzlibrary{babel}					
\usetikzlibrary{intersections, calc, shapes.geometric}		%Permite el uso de líneas guía y sus intersecciones 
\usepackage{hyperref}										% Para referencias 
\hypersetup{
    colorlinks=true,										% Define si los \ref se colorean
    linkcolor=blue,											% Define el color de los \ref
    filecolor=magenta,									% Define el color de las referencias a documentos
    urlcolor=cyan,											% Define el color de las referencias a urls
    bookmarks=true,											% Enlaza la tabla de contenidos en el visor de pdf
    pdftitle={Apuntes de Desarrollo Web},		% Es el título del pdf que recibe el visor
    breaklinks=true,										% Permite que los links ocupen más de un renglón
    linktocpage=true										% Hace que las páginas, en vez del título de las secciones sean enlazadas
}
\urlstyle{same}

% Definición del estilo de los códigos
\lstdefinestyle{mystyle}{
    backgroundcolor=\color{backcolour},   
    commentstyle=\color{codegreen},
    keywordstyle=\color{magenta},
    numberstyle=\tiny\color{codegray},
    stringstyle=\color{codepurple},
    basicstyle=\ttfamily\footnotesize,
    breakatwhitespace=false,         
    breaklines=true,                 
    captionpos=b,                    
    keepspaces=true,                 
    numbers=left,                    
    numbersep=5pt,                  
    showspaces=false,                
    showstringspaces=false,
    showtabs=false,                  
    tabsize=2
}
\lstdefinestyle{table}{
    basicstyle=\ttfamily\footnotesize,
    breakatwhitespace=false,         
    breaklines=true,              
    tabsize=2
}

\renewcommand{\familydefault}{\sfdefault}	% Fuente
% Márgenes
\addtolength{\oddsidemargin}{-.875in}
\addtolength{\evensidemargin}{-.875in}
\addtolength{\textwidth}{1.75in}

\addtolength{\topmargin}{-.875in}
\addtolength{\textheight}{1.75in}

\begin{document}
\maketitle
\begin{longtable}{c >{\ttfamily} c c >{\ttfamily\footnotesize} c}
  \caption{Etiquetas de HTML}\\
  % Encabezado que se repite
  Tipo	&	Etiqueta	&	Uso(s)	&	Código	\\\hline
  \endfirsthead
  % Para marcar continuación
  \multicolumn{4}{c}{\tablename\ \thetable{} -- continuación de la pagina anterior}\\
  Tipo	&	Etiqueta	&	Uso(s)	&	Código	\\\hline
  \endhead
  % Pie de tabla 
  \hline\multicolumn{4}{c}{Continuación en la siguiente pagina}\\\hline
  \endfoot
  % Último pie de tabla
  \hline
  \endlastfoot
  % Contenido
  % Etiquetas de texto
  \multirow{7}{*}{\rotatebox[origin=c]{90}{De texto}}
    &	\href{https://developer.mozilla.org/es/docs/Web/HTML/Element/Heading_Elements}{$<$h1$>$}
      &	Encabezado 1	&	$<$h1$>\cdots<$/h1$>$	\\
    &	\href{https://developer.mozilla.org/es/docs/Web/HTML/Element/Heading_Elements}{$<$h2$>$}
      &	Encabezado 2	& $<$h2$>\cdots<$/h2$>$	\\
    &	\href{https://developer.mozilla.org/es/docs/Web/HTML/Element/Heading_Elements}{$<$h3$>$}
      &	Encabezado 3	& $<$h3$>\cdots<$/h3$>$	\\
    &	\href{https://developer.mozilla.org/es/docs/Web/HTML/Element/Heading_Elements}{$<$h4$>$}
      &	Encabezado 4	& $<$h4$>\cdots<$/h4$>$	\\
    &	\href{https://developer.mozilla.org/es/docs/Web/HTML/Element/Heading_Elements}{$<$h5$>$}
      &	Encabezado 5	& $<$h5$>\cdots<$/h5$>$	\\
    &	\href{https://developer.mozilla.org/es/docs/Web/HTML/Element/Heading_Elements}{$<$h6$>$}
      &	Encabezado 6	& $<$h6$>\cdots<$/h6$>$	\\
    &	\href{https://developer.mozilla.org/es/docs/Web/HTML/Element/p}{$<$p$>$}
      &	Párrafo				&	$<$p$>\cdots<$/p$>$	\\
    &	\href{https://developer.mozilla.org/es/docs/Web/HTML/Element/a}{$<$a$>$}
      &	Hipervínculos	&	$<$a$>\cdots<$/a$>$	\\
  \hline
  \multirow{3}{*}{\rotatebox[origin=c]{90}{Vacíos}}
    &	\href{https://developer.mozilla.org/es/docs/Web/HTML/Element/img}{$<$img$>$}
      &	Imagen						&	$<$img src=\say{$\cdots$} alt=\say{$\cdots$} $>$\\
    &	\href{https://developer.mozilla.org/es/docs/Web/HTML/Element/br}{$<$br$>$}
      &	Salto de línea		&	$<$br$>$	\\
    &	\href{https://developer.mozilla.org/es/docs/Web/HTML/Element/hr}{$<$hr$>$}
      &	Cambio de sección	&	$<$hr$>$	\\
  \hline
  \multirow{3}{*}{\rotatebox[origin=c]{90}{Listas}}
    &	\href{https://developer.mozilla.org/es/docs/Web/HTML/Element/ol}{$<$ol$>$}
      &	Lista ordenada		&	
      \begin{lstlisting}[language=HTML]
<ol>
  <li>...</li>
</ol>
      \end{lstlisting}
      \\
    &	\href{https://developer.mozilla.org/es/docs/Web/HTML/Element/ul}{$<$ul$>$}
      &	Lista desordenada	&	
      \begin{lstlisting}[language=HTML]
<ul>
  <li>...</li>
</ul>
      \end{lstlisting}														
      \\
    &	\href{https://developer.mozilla.org/es/docs/Web/HTML/Element/hr}{$<$li$>$}
      &	Elemento de lista	&	$<$li$>\cdots<$li$>$	\\
  \hline
  \multirow{6}{*}{\rotatebox[origin=c]{90}{Tablas}}
    &	\href{https://developer.mozilla.org/es/docs/Web/HTML/Element/table}{$<$table$>$}
      &	Tabla	&
      \multirow{6}{*}{
        % \begin{frame}[fragile]
        Véase el código \ref{ls:tabla}
      % \end{frame}
      }\\
    &	\href{https://developer.mozilla.org/es/docs/Web/HTML/Element/tr}{$<$tr$>$}
      &	Fila	&	\\
    &	\href{https://developer.mozilla.org/en-US/docs/Web/HTML/Element/thead}{$<$thead$>$}
      &	Encabezado de tabla	&	\\
    &	\href{https://developer.mozilla.org/es/docs/Web/HTML/Element/th}{$<$th$>$}
      &	Celda de encabezado de tabla	&	\\
    &	\href{https://developer.mozilla.org/en-US/docs/Web/HTML/Element/tbody}{$<$tbody$>$}
      &	Cuerpo de tabla	&	\\
    &	\href{https://developer.mozilla.org/es/docs/Web/HTML/Element/td}{$<$td$>$}
      &	Celda de tabla	&	\\
  \hline
  \multirow{6}{*}{\rotatebox[origin=c]{90}{Formato}}
    &	\href{https://developer.mozilla.org/es/docs/Web/HTML/Element/strong}{$<$strong$>$}
      &	Mucho énfasis	&	$<$strong$>\cdots<$/strong$>$	\\
    &	\href{https://developer.mozilla.org/es/docs/Web/HTML/Element/i}{$<$i$>$}
      &	Itálicas	&	$<$i$>\cdots<$/i$>$	\\
    &	\href{https://developer.mozilla.org/es/docs/Web/HTML/Element/sup}{$<$sup$>$}
      &	Superíndice	&	$<$sup$>\cdots<$/sup$>$	\\
    &	\href{https://developer.mozilla.org/es/docs/Web/HTML/Element/sub}{$<$sub$>$}
      &	Subíndice	&	$<$sub$>\cdots<$/sub$>$	\\
    &	\href{https://developer.mozilla.org/es/docs/Web/HTML/Element/blockquote}{$<$blockquote$>$}
      &	Citas de bloque	&	$<$blockquote$>\cdots<$/blockquote$>$	\\
    &	\href{https://developer.mozilla.org/es/docs/Web/HTML/Element/code}{$<$code$>$}
      &	Código	&	$<$code$>\cdots<$/code$>$	\\
  \hline
  \multirow{7}{*}{\rotatebox[origin=c]{90}{Formularios}}
    &	\href{https://developer.mozilla.org/es/docs/Web/HTML/Element/form}{$<$form$>$}
      &	Formularios	&	\multirow{7}{*}{Véase el código \ref{ls:form}}	\\
    &	\href{https://developer.mozilla.org/es/docs/Web/HTML/Element/input}{$<$input$>$}	
      &	Recepción de datos	&		\\
    &	\href{https://developer.mozilla.org/es/docs/Web/HTML/Element/label}{$<$label$>$}	
      &	Etiqueta asociada a un \textit{input}	&		\\
    &	\href{https://developer.mozilla.org/es/docs/Web/HTML/Element/button}{$<$button$>$}	
      &	Botón	&		\\
    &	\href{https://developer.mozilla.org/es/docs/Web/HTML/Element/textarea}{$<$textarea$>$}	
      &	Recepción de texto multilínea	&		\\
    &	\href{https://developer.mozilla.org/es/docs/Web/HTML/Element/fieldset}{$<$fieldset$>$}	
      &	Organiza campos en un formulario	&		\\
    &	\href{https://developer.mozilla.org/es/docs/Web/HTML/Element/legend}{$<$legend$>$}	
      &	Título para \textit{fieldset}	&		\\
  \hline
  \multirow{3}{*}{\rotatebox[origin=c]{90}{\tiny Multimedia}}
    &	\href{https://developer.mozilla.org/es/docs/Web/HTML/Element/audio}{$<$audio$>$}	
      &	Insertar audio	&	$<$audio src=\say{}$>$Error$<$/audio$>$	\\
    &	\href{https://developer.mozilla.org/es/docs/Web/HTML/Element/video}{$<$video$>$}	
      &	Insertar video	&	$<$video src=\say{}$>$Error$<$/video$>$	\\
    &	\href{https://developer.mozilla.org/es/docs/Web/HTML/Element/embed}{$<$embed$>$}	
      &	Insertar contenido externo	&	$<$embed src=\say{} type=\say{}$>$	\\
  \hline
  \multirow{3}{*}{\rotatebox[origin=c]{90}{Head}}
    &	\href{https://developer.mozilla.org/es/docs/Web/HTML/Element/meta}{$<$meta$>$}	
      &	Metadatos	&	$<$meta lang=\say{es}$>$	\\
    &	\href{https://developer.mozilla.org/es/docs/Web/HTML/Element/title}{$<$title$>$}	
      &	Nombre de la pestaña	&	$<$title$>\cdots<$/title$>$	\\
    &	\href{https://developer.mozilla.org/es/docs/Web/HTML/Element/link}{$<$link$>$}	
      &	Relación con un recurso externo	&	$<$link href=\say{} rel=\say{}$>$	\\
    % &	\href{}{$<$$>$}	
    %		&		&	$<$$>\cdots<$/$>$	\\
\end{longtable}
\lstset{style=mystyle}
\begin{lstlisting}[language=HTML, caption={Tablas en HTML}, label={ls:tabla}]
<table>
  <thead>
    <tr>
      <th>Encabezado</th>
    </tr>
  </thead>
  <tbody>
    <tr>
      <td>Celda<td>
    </tr>
  </tbody>
</table>
\end{lstlisting}
\begin{lstlisting}[language=HTML, caption={Formularios en HTML}, label={ls:form}]
<form action="" method="" target="">
  <label for="x">Escribe:</label>
  <input id="x" type="" name="">

  <input type="submit" value="Enviar">
</form>\end{lstlisting}
\end{document}