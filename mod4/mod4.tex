\documentclass[12pt]{report}
\begin{document}
	\section{Teoría del diseño}
		\subsection{Composición y armonía}
			\subsubsection{Márgenes}
				\paragraph{Teoría del grid}
				\paragraph{Regla de tercios}
			\subsubsection{Balance}
				\paragraph{Simétrico}
				\paragraph{Asimétrico}
			\subsubsection{Unidad}
				\paragraph{Concepto}
			\subsubsection{Principios}
				\paragraph{Proximidad}
				\paragraph{Repetición}
				\paragraph{Énfasis}
					\begin{enumerate}
						\item Posición
						\item Continuidad
						\item Aislamiento
						\item Contraste
						\item Proporción
					\end{enumerate}
		\subsection{Colores}
		\subsection{Psicología del color}
		\subsection{Temperatura del color}
		\subsection{Valor cromático}
		\subsection{Saturación}
		\subsection{Teoría del color}
			\paragraph{Modelos}
		\subsection{Tipos de esquemas de color y sus usos}
			\paragraph{Monocromática}
			\paragraph{Análoga}
			\paragraph{Complementaria}
			\paragraph{Triádico}
			\paragraph{Split-complementario}
			\paragraph{Tetrádico}
			\paragraph{Cariantes}
		\subsection{Formato de colores} % Checar si sí se va a poner ahí
		\subsection{Crear una paleta}
			\paragraph{Formato de colores}
			\paragraph{Paletton}
			\paragraph{Colormind}
			\paragraph{AdobeColor}
		\subsection{Contraste de texto}
		\subsection{Tipografía}
			\paragraph{Tipos}
			\paragraph{Características}
			\paragraph{Legibilidad}
			\paragraph{¿Cómo combinar fuentes?}
			\paragraph{Tamaños convencionales}
	\section{CSS}
		\subsection{Declaración de CSS}
		\subsection{Agregando CSS al HTML}
		\subsection{Herencia, padres e hijos}
		\subsection{La cascada}
		\subsection{Propiedades}
			\subsubsection{Fondo}
			\subsubsection{Color}
		\subsection{Etiquetas de organización y genéricas}
		\subsection{Formas de aplicar estilos}
			\subsubsection{En línea}
			\subsubsection{Cabeza}
			\subsubsection{Archivo}
		\subsection{Selectores}
			\subsubsection{Etiqueta}
			\subsubsection{Clases}
			\subsubsection{Id}
			\subsubsection{Descendientes}
			\subsubsection{Universales}
			\subsubsection{Atributos}
			\subsubsection{Pseudoclases}
			\subsubsection{Pseudoelementos}
			\subsubsection{Combinadores}
		\subsection{Unidades}
			\subsubsection{Absolutas}
			\subsubsection{Relativas}
			\subsubsection{Propias}
		\subsection{Ordenar el contenido de un archivo CSS}
		\subsection{Formato de texto}
			\subsubsection{Fuentes}
			\subsubsection{Stack de Fuentes}
			\subsubsection{Otras propiedades}
				\paragraph{Alineación}
				\paragraph{Acomodo de texto}
		\subsection{Incrustar fuentes}
			\subsubsection{Google Fonts}
			\subsubsection{Fonts.com}
			\subsubsection{Font Squirrel}
			\subsubsection{Brick}
			\subsubsection{Font Library}
			\subsubsection{DaFont.com}
			\subsubsection{Filter}
		\subsection{Bordes}
		\subsection{Imágenes de fondo y borde}
		\subsection{Tablas}
		\subsection{Listas}
		\subsection{Enlaces}
		\subsection{Cursores}
		\subsection{Modelo de caja}
		\subsection{Propiedad posición}
			\subsubsection{Absoluta}
			\subsubsection{Relativa}
			\subsubsection{Sticky}
			\subsubsection{Fixed}
			\subsubsection{Propiedad z-index}
		\subsection{Degradados}
		\subsection{Algunas otras propiedades útiles}
			% maybe long table con nombre (Opacity, Display, Clip-path, Overflow, Máscaras, 
			% Recortes, shape-outside), descripción, caso de uso, y resultado
		\subsection{Favicon}
		\subsection{Font Awesome}
		\subsection{Transformaciones}
		\subsection{Transiciones}
		\subsection{Keyframes}
	\section{Implementar un diseño funcional y estético}
		\subsection{Diferencia entre UX, UI, diX}
		\subsection{Proceso del diseño}
		\subsection{Definición de buen diseño}
		\subsection{Aspectos a considerar}
			\subsubsection{Navegación intuitiva}
			\subsubsection{Coherencia entre páginas}
			\subsubsection{Espaciado}
			\subsubsection{Accesibilidad}
				\paragraph{Definición}
				\paragraph{Principios básicos}
					\begin{enumerate}
						\item Percepción
						\item operabilidad
						\item Inteligibilidad
						\item Robustez
					\end{enumerate}
				\paragraph{Programando páginas accesibles}
					\begin{enumerate}
						\item \textbf{Imágenes}
						\item \textbf{encabezadosCambiar tamaño a textos}
						\item \textbf{Operabilidad con teclado}
						\item \textbf{Formularios, \textit{labels}, y errores}
						\item \textbf{Contenido en movimiento, relampagueante o parpadeante}
					\end{enumerate}
				\paragraph{Evaluando la accesibilidad en un sitio web}
		\subsection{Wireframes}
			\subsubsection{Definición}
			\subsubsection{Layout Grids}
		\subsection{Layouts comunes}
		\subsection{Tendencias}
			\subsubsection{Sitios donde pueden inspirarse}
				% Table con nombre del sitio hiperreferenciado al sitio, descripción y tal vez imagen
	\section{Patrones de diseño}
		\subsection{Concepto de patrón de diseño}
			\subsubsection{Nombre}
			\subsubsection{Problema}
			\subsubsection{Contexto}
		\subsection{identificación de patrones de diseño}
			\subsubsection{Breadcrumbs}
			\subsubsection{Thumbnails}
			\subsubsection{Dropdowns}
			\subsubsection{Filtros de búsqueda}
		\subsection{Implementación}
	\section{Diseño responsivo}
		\subsection{Flex}
		\subsection{Grid}
		\subsection{Concepto de responsive}
		\subsection{Proceso y aspectos a considerar}
		\subsection{Media queries}
			\subsubsection{Link con media queries}
			\subsubsection{Media queries en CSS}
				\paragraph{Breakpoints}
	\section{Bootstrap}	
		\subsection{Concepto de framework}
		\subsection{Más usados} % ¿Qué más usado?
		\subsection{Bootstrap 5}
			\subsubsection{Convenciones}
			\subsubsection{Layout (Grid)}
			\subsubsection{Formularios}
			\subsubsection{Componentes}
\end{document}







































