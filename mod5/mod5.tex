\documentclass[12pt]{report}
\begin{document}
	\section{Herramientas}
		\subsection{IDEs}
			\subsubsection{Concepto}
			\subsubsection{Recomendaciones}
				% table con nombre del IDE con el enlace de descarga hiperreferenciado, 
				% breve descripción, ventajas, desventajas, y tal vez imagen
			\subsubsection{Instalación de XAMPP y MAMP}
				\paragraph{XAMPP}
				\paragraph{MAMP}
		\subsection{Prototipado}
			\subsubsection{Figma}
				\paragraph{¿Qué es y para qué sirve?}
				\paragraph{Menú}
					\begin{enumerate}
						\item Herramientas
						\item ComparacionesPropiedades
					\end{enumerate}
				\paragraph{Herramientas básicas}
					\begin{enumerate}
						\item Frames
						\item FigurasTexto
						\item Pluma
					\end{enumerate}
				\paragraph{Propiedades y capas}
					\begin{enumerate}
						\item Orden
						\item Color
						\item Dimensiones
					\end{enumerate}
				\paragraph{Compartir}
			% tal vez cambiar por una tabla con nombre, descripción, ventajas y desventajas
			\subsubsection{Adobe XD}
			\subsubsection{Sketch}
			\subsubsection{Justinmind}
		\subsection{Controladores de versiones}
			\subsubsection{Git}
				\paragraph{Concepto}
				\paragraph{Creación de repositorios}
				\paragraph{Comandos}
					% table con el nombre (git + clone, status, add, commit, push), descripción, 
					% código con las opciones más importantes, caso de uso, y resultado
			\subsubsection{Páginas para el control de repositorios creados con Git}
				\paragraph{GitHub}
					\subparagraph{Uso de GitHub}
					\subparagraph{Visualización y modificación de repositorios}
					\subparagraph{Integración de git para el manejo de proyectos alojados en GitHub}
	\section{Seguridad web}
		\subsection{Tipos de ataques informáticos y cómo protegerse de ellos}
			\subsubsection{XSS}
			\subsubsection{CSRF}
			\subsubsection{Inyección SQL}
		\subsection{Validaciones}
			\subsubsection{Contraseñas seguras}
			\subsubsection{Expresiones regulares}
				\paragraph{Uso en validación de datos}
			\subsubsection{Lado del cliente (HTML,JS)}
			\subsubsection{Lado del servidor (PHP)}
		\subsection{Información y datos sensibles}
			\subsubsection{Cifrado}
				\paragraph{Definición}
				\paragraph{Tipos}
					\begin{enumerate}
						\item Cifrado simétrico
						\item Cifrado asimétrico
						\item Cifrado híbrido
					\end{enumerate}
			\subsubsection{Hasheo}
				\begin{enumerate}
					\item Definición
					\item Sal
					\item Pimienta
				\end{enumerate}
	\section{Arquitectura de la información}
		\subsection{Definición}
		\subsection{Los 3 círculos}
			\subsubsection{Contexto}
			\subsubsection{Contenido}
			\subsubsection{Usuarios}
		\subsection{Encontrando información}
			\subsubsection{Necesidades}
			\subsubsection{Modelos de búsqueda}
			\subsubsection{Retroalimentación}
		\subsection{Entendiendo información}
			\subsubsection{Tipologías}
			\subsubsection{\textit{Capas} del entorno de información}
			\subsubsection{Flexibilidad}
		\subsection{Top-Down AI}
		\subsection{Bottom-Up AI}
			\subsubsection{Stress-test}
		\subsection{Sistemas de Organización}
			\subsubsection{Sistemas de organización}
				\paragraph{Dificultades}
				\paragraph{Esquemas de organización}
					\subparagraph{Exactos}
					\subparagraph{Antiguos}
						\begin{enumerate}
							\item Tópicos
							\item Orientado a tareas
							\item Específico a audiencia
							\item metafóricos
							\item Híbridos
						\end{enumerate}
				\paragraph{Estructuras de organización}
					\begin{enumerate}
						\item Jerarquía
							\begin{enumerate}
								\item Diseñando jerarquías
								\item Amplitud vs. profundidad
							\end{enumerate}
						\item Modelo Relacional
						\item Hipertexto
						\item Clasificación social
					\end{enumerate}
			\subsubsection{Etiquetado de sistemas}
				\paragraph{Importancia}
				\paragraph{Variedad}
					\begin{enumerate}
						\item Enlaces
						\item Encabezados
						\item Opciones de Navegación
						\item Index
						\item Íconos
					\end{enumerate}
				\paragraph{Diseño de etiquetas}
					\begin{enumerate}
						\item Alcance
						\item Consistencia
						\item Pruebas
					\end{enumerate}
			\subsubsection{Sistemas de navegación}
				\paragraph{Flexibilidad}
				\paragraph{Tipos}
					\begin{enumerate}
						\item Locales
						\item Globales
						\item Contextuales
					\end{enumerate}
				\paragraph{Suplementarios}
				\paragraph{Personalización}
				\paragraph{Visualización}
				\paragraph{Social}
			\subsubsection{Sistemas de búsqueda}
				\paragraph{Necesidad}
				\paragraph{Anatomía}
				\paragraph{Filtros}
				\paragraph{Algoritmos}
					\subparagraph{Coincidencia de patrones}
				\paragraph{Constructores}
				\paragraph{Resultados}
				\paragraph{Autocompletado}
				\paragraph{Búsqueda Avanzada}
			\subsubsection{Implementación}
				\paragraph{Investigación}
				\paragraph{Estrategias}
				\paragraph{Diseño}
	\section{Técnicas de optimización para reducir tiempo de carga en sitios}
		\subsection{Renderizado de contenido por navegadores}
		\subsection{Optimización JS}
			\paragraph{Optimización manual}
			\paragraph{Minifiers}
			\paragraph{Funciones inline}
			\paragraph{DOM}
			\paragraph{Delegación de eventos}
			\paragraph{Ciclos}
			\paragraph{Scope}
			\paragraph{Async}
		\subsection{Optimización CSS}
			\paragraph{Propiedades abreviadas}
			\paragraph{Agrupación de selectores}
			\paragraph{Selectores anidados}
			\paragraph{Minifiers}
		\subsection{Optimización HTML}
			\paragraph{Comentarios y espacios}
			\paragraph{CSS y JS de línea}
			\paragraph{Atributos Booleanos}
		\subsection{Optimización de imágenes}
			\paragraph{Formatos}
			\paragraph{Optimización por zona}
			\paragraph{Lazy Loadinf}
			\paragraph{Renderizado de imágenes}
			\paragraph{Data URIs}
			\paragraph{Ancho y Alto}
			\paragraph{Mapeo de imágenes}
		\subsection{Otros tips}
			\paragraph{CSS en head}
		\subsection{Optimización PHP}
			\paragraph{Optimización menor}
			\paragraph{Optimización mayor}
				\begin{enumerate}
					\item Ciclos
					\item Regex
					\item Incluir archivos
				\end{enumerate}
		\subsection{Lighthouse}
\end{document}







































