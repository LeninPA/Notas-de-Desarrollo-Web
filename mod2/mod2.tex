\documentclass[12pt]{report}
\begin{document}
	% \section{Conceptos fundamentales}
	% 	\subsection{¿Dato o información?}
	% 		% table con la definición y diferencias entre ellas
	% 	\subsection{Tipos de bases de datos}
	% 		\subsubsection{Por su ubicación}
	% 			\paragraph{Centralizadas}
	% 			\paragraph{Distribuidas}
	% 		\subsubsection{Por la estructura}
	% 			\paragraph{Estructurada}
	% 			\paragraph{Semiestructurada}
	% 		\subsubsection{Por el modelo}
	% 			\paragraph{Jerárquicas}
	% 			\paragraph{Relacionales}
	% 			\paragraph{Orientadas a objetos}
	% 			\paragraph{Orientadas a documentos}
	% 	\subsection{Modelo relacional}
			\paragraph{Reglas de Codd} Tomadas de 
			\begin{enumerate}[label=\textbf{Regla \arabic*.},start=0]
				\item Un sistema de gestión de bases de datos relacional debe 
					gestionar sus datos almacenados sólo con el uso de sus capacidades
					relacionales. Éste es el principio fundamental sobre el que se 
					basan las 12 reglas restantes.
				\item \textit{Representación de información.} Toda información debe 
					representarse, en el nivel lógico, sólo como valores en tablas.
				\item \textit{Acceso garantizado.} Debe ser posible acceder a 
					cualquier ítem de datos en la base de datos al proporcionar su 
					nombre de tabla, nombre de columna y valor de clave primaria.
				\item \textit{Representación de valores nulos.} El sistema debe 
					ser capaz de representar valores nulos en una forma sistemática, 
					sin importar el tipo de datos del ítem. Los valores nulos deben 
					ser distintos de cero o cualquier otro número, y de cadenas vacías.
				\item \textit{Catálogo relacional.} El catálogo del sistema, que 
					contiene la descripción lógica de la base de datos, debe 
					representarse de la misma forma que los datos ordinarios.
				\item \textit{Sublenguaje de datos amplio.} Sin importar el número 
					de otros lenguajes que soporte, la base de datos debe incluir un 
					lenguaje que permite enunciados expresados como cadenas de 
					caracteres para soportar definición de datos, definición de vistas, 
					manipulación de datos, reglas de integridad, autorización de usuario 
					y un método de identificación de unidades para recuperación.
				\item \textit{Actualización de vistas.} Cualquier vista que sea 
					teóricamente actualizable en realidad la puede actualizar el sistema
				\item \textit{Operaciones Insert, Delete y Update.} Cualquier relación
					que se pueda manejar como un solo operando para recuperación 
					(retrieval) también se puede manejar de esa forma para operaciones 
					de inserción, borrado y actualización.
				\item \textit{Independencia física de datos.} Los programas de aplicación 
					son inmunes a cambios hechos a representaciones de almacenamiento o 
					métodos de acceso.
				\item \textit{Independencia lógica de datos.} Los cambios efectuados 
					a nivel lógico, como dividir tablas o combinar tablas, que no afectan 
					el contenido de información a nivel lógico, no requieren modificación 
					de aplicaciones.
				\item \textit{Reglas de integridad.} Las restricciones de integridad 
					como la integridad de entidad y la integridad referencial deben 
					especificarse en el sublenguaje de datos y almacenarse en el 
					catálogo. Para expresar estas restricciones no se deben usar enunciados 
					de programa de aplicación.
				\item \textit{Independencia de distribución.} El sublenguaje de datos 
					debe ser tal que, si la base de datos se distribuye, los programas de 
					aplicación y los comandos de los usuarios no necesitan cambiar.
				\item \textit{No subversión.} Si el sistema permite un lenguaje 
					que soporte acceso a registro a la vez, cualquier programa que use 
					este tipo de acceso no puede pasar por alto las restricciones de 
					integridad expresadas en el lenguaje de nivel superior.
			\end{enumerate}
	% 	\subsection{Características de las bases de datos}
	% 		\begin{enumerate}
	% 			\item Redundancia mínima
	% 			\item Integridad
	% 			\item Independencia lógica y física de los datos
	% 		\end{enumerate}
	% \section{Planificación del proyecto y definición del sistema}
	% 	\subsection{Diseño de la base de datos}
	% 		\subsubsection{Pasos del diseño} % Cambiar para ser más exacto
	% 			\paragraph{Diseño conceptual}
	% 			\paragraph{Diseño lógico}
	% 			\paragraph{Diseño físico}
	% 		\subsubsection{Llaves}
	% 			% table con el nombre de las llaves definición y función
	% 		\subsubsection{Diccionario de datos}
	% 			\paragraph{Definición}
	% 			% ejemplo chiquito
	% 		\subsubsection{Diagrama entidad-relación}
	% 			\paragraph{Elementos del modelo}
	% 				\begin{enumerate}
	% 					\item Entidades
	% 					\item Atributos
	% 					\item Relaciones
	% 				\end{enumerate}
	% 			\paragraph{Representación gráfica}
	% 			\paragraph{Cardinalidad}
	% 			\paragraph{Tipos de relaciones}
	% 				\begin{enumerate}
	% 					\item Tipo 1
	% 				\end{enumerate}
	% 			\paragraph{Modelo Chen}
	% 			\paragraph{UML}
	% 	\subsection{Construcción, implementación y mantenimiento}
	% 		% Pendiente ver qué va aquí
	% 	\subsection{Normalización}
	% 		\subsubsection{Primera forma normal (1NF)}
	% 		\subsubsection{Segunda forma normal (2NF)}
	% 		\subsubsection{Tercera forma normal (3NF)}
	% 	\subsection{Sistemas Manejadores de Bases de datos (DBMS)}
		\subsection{Lenguaje de Consulta Estructurado (SQL)}
			\subsubsection{Tipos de datos}
					\begin{center}
						\begin{longtable}{c m{2.5cm} m{4cm} m{5cm}}
							\caption{Etiquetas de HTML}\\
							% Encabezado que se repite
							\textbf{Tipo}	&	\textbf{Nombre}	&	\textbf{Descripción}	&	\textbf{Dominio}\\[0.5cm]\hline
							\endfirsthead
							% Para marcar continuación
							\multicolumn{4}{c}{\tablename\ \thetable{} -- continuación de la pagina anterior}\\
							\textbf{Tipo}	&	\textbf{Nombre}	&	\textbf{Descripción}	&	\textbf{Dominio}\\[0.5cm]\hline
							\endhead
							% Pie de tabla 
							\hline\multicolumn{4}{c}{Continuación en la siguiente pagina}\\\hline
							\endfoot
							% Último pie de tabla
							\hline
							\endlastfoot
							% Contenido
							
							\multirow{9}{*}{\rotatebox[origin=c]{90}{Numéricos}}
								&	BOOL o\newline BOOLEAN	&	Valor de falso o verdadero		&	0, 1	\\[0.7cm]
								&	BIT							&	Un bit												&	1 a 64\\[0.7cm]
								&	TINYINT					&	Número entero diminuto con o sin signo	&	-128 a 127\newline0 a 255\\[0.7cm]
								&	SMALLINT				&	Número entero chiquito con o sin signo	&	-32768 a 32767\newline0 a 65535\\[0.7cm]
								&	MEDIUMINT				&	Número entero mediano con o sin signo		&	-8388608 a 8388607\newline0 a 65535\\[0.7cm]
								&	INT\newline o INTEGER		&	Número entero con o sin signo						&	-2147483648 a 2147483647\newline0 a 4294967295\\[0.7cm]
								&	BIGINT					& Número entero grande con o sin signo		&	-9223372036854775808 a 9223372036854775807\newline0 a 18446744073709551615\\[0.7cm]
								& DECIMAL\newline(\textit{tam,d})	& Número decimal con un número de dígitos predefinido	&	Depende del atributo \textit{tam}\\[0.7cm]
								&	FLOAT\newline(\textit{tam,d})		&	Número flotante pequeño					&	$-3.402823466\times10^{38}$ a $-1.175494351\times10^{38}$ ó\newline
										$0$ ó\newline
										$-1.175494351\times10^{38}$ a $-3.402823466\times10^{38}$
									\\	
							\multirow{4}{*}{\rotatebox[origin=c]{90}{Texto}}
								&	CHAR(\textit{tam})		&	Una cadena de texto de longitud \textbf{fija}	de 0 a 255 caracteres	&	Depende de la base de datos\\
								&	VARCHAR(\textit{tam})	&	Una cadena de texto de longitud \textbf{variable}	de 0 a 255 caracteres	&	Depende de la base de datos\\
								&	TINYBLOB							&	Se usa para guardar \href{https://es.wikipedia.org/wiki/Binary_large_object}{BLOBs} de hasta 255 caracteres	&	Depende de la base de datos\\
								&	TEXT									&	Se usa para guardar hasta 65,535 caracteres de texto	&	Depende de la base de datos\\
							\multirow{4}{*}{\rotatebox[origin=c]{90}{Fecha}}
								&	TIME			&	Una hora de formato hh:mm:ss												&	De '-838:59:59' a '838:59:59'\\
								&	DATE			&	Una fecha de formato AAAA-MM-DD											&	De '1000-01-01' a '9999-12-31'\\
								&	DATETIME	&	Una fecha y una hora de formato AAAA-MM-DD hh:mm:ss	&	De '1000-01-01 00:00:00' a '9999-12-31 23:59:59'\\
								&	YEAR			&	Un año de formato AAAA															&	De 1901 a 2155, y 0000\\
								&	TIMESTAMP	&	Atributo en el que se almacena el tiempo UNIX				&	De '1970-01-01 00:00:01' UTC a '2038-01-09 03:14:07' UTC
						\end{longtable}
					\end{center}
	% 		\subsubsection{Lenguaje de Definición de Datos (DDL)}
	% 			\paragraph{Creación de bases de datos y de tablas}
	% 			\paragraph{Modificación de tablas}
	% 			\paragraph{Eliminación de bases de datos y de tablas}
	% 		\subsubsection{Lenguaje de Manipulación de Datos (DML)}
	% 			\paragraph{Inserción de datos}
	% 			\paragraph{Restricciones (\textit{clauses})}
	% 			\paragraph{Modificación de registros}
	% 			\paragraph{Eliminación de registros}
	% 		\subsubsection{Lenguaje de Manipulación de Datos (DML)}
	% 			\paragraph{Creación y elminación de usuarios}
	% 			\paragraph{Creación y elminación de permisos}
	% 		\subsubsection{Consultas simples}
	% 			\paragraph{Comparaciones}
	% 			\paragraph{Comparación con rango}
	% 			\paragraph{Test de pertenencia}
	% 			\paragraph{Comparación con patrones}
	% 			\paragraph{Comparación con valores nulos}
	% 		\subsubsection{Funciones de agrupamiento \textit{aggregate functions}}
	% 			\paragraph{Promedio}
	% 			\paragraph{Cuenta}
	% 			\paragraph{Mínimo y máximo}
	% 			\paragraph{Suma}
	% 		\subsubsection{Uniones}
	% 			\paragraph{Natural join}
	% 			\paragraph{Inner join}
	% 			\paragraph{Outer join}
	% 			\paragraph{Producto cardinal}
	% 	\subsection{Integración de consultas SQL en págunas programadas en lenguaje PHP}
	% 		\subsubsection{CRUD (Crear, Leer, Actualizar y Borrar)}
	% 			\paragraph{Definición}
	% 			% table con cada función, código, y maybe ejemplo
	% 		\subsubsection{Manipulación de respuestas a una consulta}
	% 		\subsubsection{Errores de conexión a la base de datos}
	% 			% table con tipos de errores
\end{document}







































