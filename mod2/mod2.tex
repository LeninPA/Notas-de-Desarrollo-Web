\documentclass[12pt]{report}
\begin{document}
	\section{Conceptos fundamentales}
		\subsection{¿Dato o información?}
			% table con la definición y diferencias entre ellas
		\subsection{Tipos de bases de datos}
			\subsubsection{Por su ubicación}
				\paragraph{Centralizadas}
				\paragraph{Distribuidas}
			\subsubsection{Por la estructura}
				\paragraph{Estructurada}
				\paragraph{Semiestructurada}
			\subsubsection{Por el modelo}
				\paragraph{Jerárquicas}
				\paragraph{Relacionales}
				\paragraph{Orientadas a objetos}
				\paragraph{Orientadas a documentos}
		\subsection{Modelo relacional}
			\paragraph{Reglas de Codd} Es esto
			\begin{enumerate}
				\item regla 1
			\end{enumerate}
		\subsection{Características de las bases de datos}
			\begin{enumerate}
				\item Redundancia mínima
				\item Integridad
				\item Independencia lógica y física de los datos
			\end{enumerate}
	\section{Planificación del proyecto y definición del sistema}
		\subsection{Diseño de la base de datos}
			\subsubsection{Pasos del diseño} % Cambiar para ser más exacto
				\paragraph{Diseño conceptual}
				\paragraph{Diseño lógico}
				\paragraph{Diseño físico}
			\subsubsection{Llaves}
				% table con el nombre de las llaves definición y función
			\subsubsection{Diccionario de datos}
				\paragraph{Definición}
				% ejemplo chiquito
			\subsubsection{Diagrama entidad-relación}
				\paragraph{Elementos del modelo}
					\begin{enumerate}
						\item Entidades
						\item Atributos
						\item Relaciones
					\end{enumerate}
				\paragraph{Representación gráfica}
				\paragraph{Cardinalidad}
				\paragraph{Tipos de relaciones}
					\begin{enumerate}
						\item Tipo 1
					\end{enumerate}
				\paragraph{Modelo Chen}
				\paragraph{UML}
		\subsection{Construcción, implementación y mantenimiento}
			% Pendiente ver qué va aquí
		\subsection{Normalización}
			\subsubsection{Primera forma normal (1NF)}
			\subsubsection{Segunda forma normal (2NF)}
			\subsubsection{Tercera forma normal (3NF)}
		\subsection{Sistemas Manejadores de Bases de datos (DBMS)}
		\subsection{Lenguaje de Consulta Estructurado (SQL)}
			\subsubsection{Tipos de datos}
				% tabla con el nombre, descripción y ejemplo
			\subsubsection{Lenguaje de Definición de Datos (DDL)}
				\paragraph{Creación de bases de datos y de tablas}
				\paragraph{Modificación de tablas}
				\paragraph{Eliminación de bases de datos y de tablas}
			\subsubsection{Lenguaje de Manipulación de Datos (DML)}
				\paragraph{Inserción de datos}
				\paragraph{Restricciones (\textit{clauses})}
				\paragraph{Modificación de registros}
				\paragraph{Eliminación de registros}
			\subsubsection{Lenguaje de Manipulación de Datos (DML)}
				\paragraph{Creación y elminación de usuarios}
				\paragraph{Creación y elminación de permisos}
			\subsubsection{Consultas simples}
				\paragraph{Comparaciones}
				\paragraph{Comparación con rango}
				\paragraph{Test de pertenencia}
				\paragraph{Comparación con patrones}
				\paragraph{Comparación con valores nulos}
			\subsubsection{Funciones de agrupamiento \textit{aggregate functions}}
				\paragraph{Promedio}
				\paragraph{Cuenta}
				\paragraph{Mínimo y máximo}
				\paragraph{Suma}
			\subsubsection{Uniones}
				\paragraph{Natural join}
				\paragraph{Inner join}
				\paragraph{Outer join}
				\paragraph{Producto cardinal}
		\subsection{Integración de consultas SQL en págunas programadas en lenguaje PHP}
			\subsubsection{CRUD (Crear, Leer, Actualizar y Borrar)}
				\paragraph{Definición}
				% table con cada función, código, y maybe ejemplo
			\subsubsection{Manipulación de respuestas a una consulta}
			\subsubsection{Errores de conexión a la base de datos}
				% table con tipos de errores
\end{document}







































