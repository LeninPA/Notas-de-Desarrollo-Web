\documentclass[12pt]{report}
\begin{document}
	% TODO: Actualizar la tabla de contenidos
	\section{¿Qué es JavaScript?}
		\subsection{Breve historia}
		\subsection{Características del lenguaje}
	\section{Incorporar JavaScript en una página web}
		\subsection{Etiqueta $<$script$>$}
		\subsection{Atributo \textit{src} y archivos .js}
	\section{Variables y constantes}
		\subsection{Declaración de variables}
			% table con el nombre var o let, descripción, ventajas, desventajas y ejemplo
		\subsection{Reglas para el nombramiento de identificadores}
		\subsection{Tipado debil o dinámico}
		\subsection{Tipo de datos}
			% table con el tipo de dato (undefined, boolean, number, string, null),
			% descripción y ejemplo
			tabla
	\section{Operadores}
		% Long table con clasificación en vertical (De asignación, aritméticos, lógicos,
		% de comparación, de cadena, terniarios, relacionales), nombre, escritura, y caso de uso
	\section{Estructuras de control}
		% Long table con su clasificación en condicionales (if, else, if... else, switch),
		% e iterativas (for, while, do... while, for each), nombre, escritura y caso de uso
	\section{Funciones predefinidas}
		\subsection{Typeof}
			% Descripción y caso de uso
		\subsection{IsNan}
			% Descripción y caso de uso
	\section{Programación Orientada a Objetos (POO)}
		\subsection{Conceptos básicos}
			\subsubsection{Definición del Paradigma POO}
			\subsubsection{Definición de objeto}
			\subsubsection{Creación de objetos}
		\subsection{Tipos de objeto predefinidos}
			\subsubsection{Array}
				\paragraph{Propiedades} % length, y key
				\paragraph{Métodos}
					% longtable con la clasificación de la función: de manipulación de datos (push, pop
					% shift, unshift, concat, slice, splice, reverse, sort), de búsqueda (indexOf, 
					% lastIndexOf, find, findIndex), u operaciones (map, filter, reduce); nombre; 
					% código; ejemplo; y lo que devuelve
			\subsubsection{Math}
				\paragraph{Propiedades} % Pi
				\paragraph{Métodos}
					% longtable con el nombre; código; ejemplo; y lo que devuelve
			\subsubsection{String}
				\paragraph{Propiedades} % length
					propiedades
				\paragraph{Métodos}
					% longatble con la clasificación de la función: de manipulación (split, substr,
					% concat, toUpperCase, toLowerCase, replace), o de búsqueda (indexOf, lastIndexOf,
					% search, match); nombre; código; ejemplo; y lo que devuelve
		\subsection{Herencia}
			herencia
	\section{Funciones declaradas por el programador}
		buenas
		\subsection{Declaración}
		\subsection{Parámetros por defecto}
		\subsection{Llamada a funciones}
		\subsection{Referencias de funciones}
		\subsection{Funciones anónimas}
	\section{Cookies}
		\subsection{Creación de cookies}
	\section{DOM o \textit{Document Object Model}}
		\subsection{Selección de elementos}
			\subsection{Por id}
			\subsection{Por clase}
			\subsection{Por etiqueta}
			\subsection{Por query}
		\subsection{Crear elementos}
		\subsection{Añadir elementos}
	\section{Eventos}
		buenas
		\subsection{Conceptos básicos}
			\subsubsection{Definición del paradigma de Programación Orientada a Eventos}
			\subsubsection{Definición de evento}
		\subsection{Tipos de evento en JS}
		\subsection{Añadir eventos}
		\subsection{Modificar eventos}
	\section{JQuery}
		\subsection{Concepto}
		\subsection{Instalación}
		\subsection{Aplicaciones}
			\subsubsection{Simplificación del manejo del DOM}
			\subsubsection{Peticiones}
	\section{AJAX}
		\subsection{Concepto}
		\subsection{Manejo con JQuery para enciar información}
	\section{JSON}
		\subsection{Concepto}
		\subsection{Enviar información}
	\section{Canvas}
		buenas
		\subsection{Etiqueta $<$canvas$>$}
		\subsection{Etiqueta $<$canvas$>$, obtener contexto 2D}
		\subsection{Figuras}
			\subsubsection{Relleno}
			\subsubsection{Bordes}
	\section{Imágenes y sonidos}
	\section{Animaciones}
	\section{Interactividad}
\end{document}







































