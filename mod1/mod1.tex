\documentclass[12pt]{report}
\begin{document}
	% \section{Conceptos fundamentales}
	% 	\subsection{Internet vs. Web}
			
	% 	\subsection{Una breve historia del internet}
			
	% 	\subsection{Protocolo HTTP}
			
	% 	\subsection{Navegadores Web}
			
	% 	\subsection{URL}
			
	\section{HTML5}
		% \subsection{¿Qué es HTML?}
		% 	\subsubsection{Breve historia}
		% 	\subsubsection{Características}
		% 		a
		% \subsection{Estructura de un archivo HTML}
		% 	\subsubsection{¿Qué es una etiqueta?}
		% 	\subsubsection{Estructura de una etiqueta}
		% 	\subsubsection{Anatomía de una página HTML}
		% 	\subsubsection{Semántica}
		% \subsection{Comentarios}
		\subsection{Etiquetas}
			% Maybe una longtable en la que se tenga tipo/clasificación,
			% comando, código y cómo se ve en el editor
			% No olvidar poner el a, maybe?
			\begin{longtable}{c >{\ttfamily} c c >{\ttfamily\footnotesize} c}
				\caption{Etiquetas de HTML}\\
				% Encabezado que se repite
				Tipo	&	Etiqueta	&	Uso(s)	&	Código	\\\hline
				\endfirsthead
				% Para marcar continuación
				\multicolumn{4}{c}{\tablename\ \thetable{} -- continuación de la pagina anterior}\\
				Tipo	&	Etiqueta	&	Uso(s)	&	Código	\\\hline
				\endhead
				% Pie de tabla 
				\hline\multicolumn{4}{c}{Continuación en la siguiente pagina}\\\hline
				\endfoot
				% Último pie de tabla
				\endlastfoot
				% Contenido
				% Etiquetas de texto
				\multirow{7}{*}{\rotatebox[origin=c]{90}{De texto}}
					&	\href{https://developer.mozilla.org/es/docs/Web/HTML/Element/Heading_Elements}{$<$h1$>$}
						&	Encabezado 1	&	$<$h1$>\cdots<$/h1$>$	\\
					&	\href{https://developer.mozilla.org/es/docs/Web/HTML/Element/Heading_Elements}{$<$h2$>$}
						&	Encabezado 2	& $<$h2$>\cdots<$/h2$>$	\\
					&	\href{https://developer.mozilla.org/es/docs/Web/HTML/Element/Heading_Elements}{$<$h3$>$}
						&	Encabezado 3	& $<$h3$>\cdots<$/h3$>$	\\
					&	\href{https://developer.mozilla.org/es/docs/Web/HTML/Element/Heading_Elements}{$<$h4$>$}
						&	Encabezado 4	& $<$h4$>\cdots<$/h4$>$	\\
					&	\href{https://developer.mozilla.org/es/docs/Web/HTML/Element/Heading_Elements}{$<$h5$>$}
						&	Encabezado 5	& $<$h5$>\cdots<$/h5$>$	\\
					&	\href{https://developer.mozilla.org/es/docs/Web/HTML/Element/Heading_Elements}{$<$h6$>$}
						&	Encabezado 6	& $<$h6$>\cdots<$/h6$>$	\\
					&	\href{https://developer.mozilla.org/es/docs/Web/HTML/Element/p}{$<$p$>$}
						&	Párrafo				&	$<$p$>\cdots<$/p$>$	\\
				\multirow{3}{*}{\rotatebox[origin=c]{90}{Vacíos}}
					&	\href{https://developer.mozilla.org/es/docs/Web/HTML/Element/img}{$<$img$>$}
						&	Imagen						&	$<$img src=\say{$\cdots$} alt=\say{$\cdots$} $>$\\
					&	\href{https://developer.mozilla.org/es/docs/Web/HTML/Element/br}{$<$br$>$}
						&	Salto de línea		&	$<$br$>$	\\
					&	\href{https://developer.mozilla.org/es/docs/Web/HTML/Element/hr}{$<$hr$>$}
						&	Cambio de sección	&	$<$hr$>$	\\
				\multirow{3}{*}{\rotatebox[origin=c]{90}{Listas}}
					&	\href{https://developer.mozilla.org/es/docs/Web/HTML/Element/ol}{$<$ol$>$}
						&	Lista ordenada		&	
						\begin{lstlisting}[language=HTML]
<ol>
	<li>...</li>
</ol>
						\end{lstlisting}
						\\
					&	\href{https://developer.mozilla.org/es/docs/Web/HTML/Element/ul}{$<$ul$>$}
						&	Lista desordenada	&	
						\begin{lstlisting}[language=HTML]
<ul>
	<li>...</li>
</ul>
						\end{lstlisting}														
						\\
					&	\href{https://developer.mozilla.org/es/docs/Web/HTML/Element/hr}{$<$li$>$}
						&	Elemento de lista	&	$<$li$>\cdots<$li$>$	\\
				\multirow{6}{*}{\rotatebox[origin=c]{90}{Tablas}}
					&	\href{https://developer.mozilla.org/es/docs/Web/HTML/Element/table}{$<$table$>$}
						&	Tabla	&
						\multirow{6}{*}{
							% \begin{frame}[fragile]
							Véase el código \ref{ls:tabla}
						% \end{frame}
						}\\
					&	\href{https://developer.mozilla.org/es/docs/Web/HTML/Element/tr}{$<$tr$>$}
						&	Fila	&	\\
					&	\href{https://developer.mozilla.org/en-US/docs/Web/HTML/Element/thead}{$<$thead$>$}*
						&	Encabezado de tabla	&	\\
					&	\href{https://developer.mozilla.org/en-US/docs/Web/HTML/Element/thead}{$<$th$>$}
						&	Celda de encabezado de tabla	&	\\
					&	\href{https://developer.mozilla.org/en-US/docs/Web/HTML/Element/thead}{$<$tbody$>$}
						&	Cuerpo de tabla	&	\\
					&	\href{https://developer.mozilla.org/en-US/docs/Web/HTML/Element/thead}{$<$td$>$}
						&	Celda de tabla	&	\\
					% &	\href{}{$<$$>$}	
					%		&		&	$<$$>\cdots<$/$>$	\\
			\end{longtable}
			*: En inglés
			\lstset{style=mystyle}
			\begin{lstlisting}[language=HTML, caption={Tablas en HTML}, label={ls:tabla}]
<table>
	<thead>
		<tr>
			<th>Encabezado</th>
		</tr>
	</thead>
	<tbody>
		<tr>
			<td>Celda<td>
		</tr>
	</tbody>
</table>
			\end{lstlisting}

	% 	\subsection{Hipervínculos y rutas}
	% 	\subsection{Imágenes}
	% 		\subsubsection{Formatos}
	% 		\subsubsection{Dimensiones}
	% 		\subsubsection{SVG}
	% 		\subsubsection{Mapeo}
	% 	\subsection{Formularios}
	% 		\subsubsection{$<$form$>$}
	% 			\paragraph{Atributo \textit{action}}
	% 			\paragraph{Atributo \textit{method}}
	% 			\paragraph{Atributo \textit{target}}
	% 		\subsubsection{$<$input$>$}
	% 			\paragraph{Atributos comunes}
	% 			\paragraph{Forma de uso}
	% 		\subsubsection{$<$select$>$}
	% 		\subsubsection{$<$button$>$}
	% 		\subsubsection{$<$label$>$}
	% 		\subsubsection{$<$textarea$>$}
	% 		\subsubsection{$<$fieldset$>$ y $<$legend$>$}
	% 	\subsection{Incrustación de archivos multimedia}
	% 		% Maybe un longtable con audio, video y embed
	% 	\subsection{Etiqueta $<$head$>$}
	% 		% Maybe un longtable con meta, title y link
	% \section{PHP 7}
	% 	\subsection{¿Qué es PHP?}
	% 		\subsubsection{Breve historia}
	% 		\subsubsection{Características del lenguaje PHP}
	% 		\subsubsection{Creación de programas PHP}
	% 	\subsection{Sintaxis}
	% 		\subsubsection{Variables y constantes}
	% 		\subsubsection{Tipos de dato}
	% 			\paragraph{Entero}
	% 			\paragraph{Punto flotante}
	% 			\paragraph{Booleano}
	% 			\paragraph{Cadena}
	% 		\subsubsection{Operadores}
	% 			% Maybe un longtable con operadores aritméticos, 
	% 			% de asignación lógicos, relacionales y de concatenación
	% 	\subsection{Funciones de salida}
	% 		% tabla con echo, print, printf y sprintf
	% 	\subsection{Estructuras de control}
	% 		% tabla dividida en condicionales(if, ifelse, switch) e 
	% 		% iterativas(for, while, dowhile, foreach)
	% 	\subsection{Funciones declaradas por el programador}
	% 		\subsubsection{Declaración}
	% 		\subsubsection{Parámetros}
	% 		\subsubsection{Valor de retorno}
	% 	\subsection{Peticiones HTTP}
	% 		% Maybe una tabla con nombre, código, ventajas y desventaas de POST, GET y FILE
	% 	\subsection{Arreglos}
	% 		\subsubsection{Declaración}
	% 		\subsubsection{Arreglos asociativos}
	% 		\subsubsection{Arreglos unidimensionales y bidimensionales}
	% 		\subsubsection{Funciones útiles}
	% 			% tabla con nombre, descripción, declaración, ejemplo, resultado
	% 	\subsection{Funciones predefinidas}
	% 		% longtable con clasificación de la función en vertical, nombre, descripciñon, declaración,
	% 		% ejemplo y resultado
	% 	\subsection{Trabajo con varios archivos PHP}
	% 		\subsubsection{Include}
	% 		\subsubsection{Require}
	% 	\subsection{Codificación de caracteres}
	% 		\subsubsection{Concepto}
	% 		\subsubsection{Tipos}
	% 	\subsection{Sesiones}
	% 		\subsubsection{Cookie vs. sesión}
	% 			% una fila con los encabezados y las diferencias una frente a otra
	% 		\subsubsection{Manejo de sesión}
	% 			\paragraph{Apertura de sesiones}
	% 			\paragraph{Cierre de sesiones}
\end{document}







































